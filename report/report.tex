\documentclass[12pt, a4paper]{article}
\usepackage[utf8]{inputenc}
\usepackage[T2A]{fontenc}
\usepackage[english, russian]{babel}
\usepackage[usenames,dvipsnames]{xcolor}
\usepackage{listings,a4wide,longtable,amsmath,amsfonts,graphicx,tikz}
\usepackage{indentfirst,verbatim}

\usepackage{minted}
\setminted[rust]{
  fontsize=\scriptsize,
  baselinestretch=1.2,
  linenos,
  frame=lines
}

\begin{document}
\thispagestyle{empty}
\begin{center}
  {\large
    Университет ИТМО \\
    Кафедра Информатики и прикладной математики \\
  }
\end{center}
\vspace{\stretch{2}}
\begin{center}
  {\large
    Проектирование языков программирования и языков представления знаний\\
  }
    \vspace{\stretch{1}}
  {\large
    Лабораторная работа 1\\
    ``Разработка лексического анализатора''\\
  }

\end{center}
\vspace{\stretch{6}}
\begin{flushright}
  Работу выполнил студент группы P4117\\
  {\it Фомин Евгений\\}
\end{flushright}
\vspace{\stretch{4}}
\begin{center}
  2018
\end{center}
\newpage

\section{Цель работы}
Разработать лексический анализатор для реализуемого языка и
проверить корректность его работы на примерах.

\section{Исходные данные}
\subsection{БНФ реализуемого языка}
\verbatiminput{bnf}

Комментарии даны в фигурных скобках.

\subsection{Список классов лексем}
\begin{description}
\item[Ключевое слово (Keyword)] Зарезервированные строки символов, например \texttt{Begin}
\item[Идентификатор (Name)] Объявленнная переменная или ее использование в программе
\item[Символ (Symbol)] Зарезервированные спецсимволы, применяемые в выражениях
\item[Число (Number)] Численная константа

\end{description}

\section{Ход работы}

\subsection{Исходный код программы с комментариями}
\inputminted{rust}{main.rs}

\subsection{Позитивный пример}
\verbatiminput{input_one}
\subsubsection{Результат}
\verbatiminput{input_one_result}

\subsubsection{Пример с некорректной лексемой}
\verbatiminput{input_two}
\subsubsection{Результат}
\verbatiminput{input_two_result}

\section{Вывод}

В ходе работы был реализован лексический анализатор для
небольшого Паскаль-подобного языка. Проверяя на допустимые лексемы,
корректность программы, составленной из них,
остается за рамками данной лабораторной
работы и является задачей синтаксического анализатора.

\end{document}
